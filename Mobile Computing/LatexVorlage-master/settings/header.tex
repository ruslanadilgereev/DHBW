% Include from Einstellungen.tex
\title{\titel}
\author{\autor}
%\date{\today} % set date for default title page (\maketitle) -- not used with custom title page

\usepackage[english, main=ngerman]{babel} % set language -- ngerman is default, english needed for acro

\usepackage{scrhack} % improves compatibility with KOMA

% handle encoding
\usepackage[T1]{fontenc} % use 256-glyph font (e.g. makes "ü" a glyph) enables copy-pasting of umlauts
\usepackage[utf8]{inputenc} % allow all UTF-8 characters in source

% page geometry
\usepackage[left=2.50cm, right=2.50cm, top=2.50cm, bottom=3.10cm]{geometry}	% define margins
\usepackage{pdflscape} % allow landscape pages
\renewcommand*{\chapterheadstartvskip}{\vspace*{.7\baselineskip}}% reduce chapter headings margin
\setcounter{secnumdepth}{3} % number subsubsections
\setcounter{tocdepth}{2} %subsections im toc anzeigen
\setuptoc{toc}{totoc} % add toc to toc

\usepackage[onehalfspacing]{setspace} % line spacing 1.5
\usepackage{microtype} % improve typesetting -> avoid over- / underfull boxes
\usepackage{textcomp} % defines additional symbols (see: https://ctan.org/pkg/textcomp ??)
% Hurenkinder und Schusterjungen verhindern
% http://projekte.dante.de/DanteFAQ/Silbentrennung -- link deprecated!
\clubpenalty = 10000 % schließt Schusterjungen aus (Seitenumbruch nach der ersten Zeile eines neuen Absatzes)
\widowpenalty = 10000 % schließt Hurenkinder aus (die letzte Zeile eines Absatzes steht auf einer neuen Seite)
\displaywidowpenalty=10000


% Font
\usepackage{noto}
%\usepackage[scaled]{helvet}
%\usepackage{lmodern}

\usepackage{etoolbox} %Untersützt ifstrempty

% footnotes
\usepackage{chngcntr} 
\counterwithout{footnote}{chapter} % make footnotes not reset at the beginning of a chapter
\renewcommand{\thempfootnote}{\arabic{mpfootnote}} % number footnotes
\usepackage[perpage, hang, multiple, stable]{footmisc} % add footnote stuff

\usepackage{afterpage} % allows control over float placement (see: https://www.ctan.org/pkg/afterpage)


% math support
\usepackage{amsmath} % math macros like \hat
\usepackage{amsfonts}
\usepackage{amssymb}

% units
\usepackage[locale=DE]{siunitx} % adds \qty and \unit
\DeclareSIUnit[qualifier-mode=combine]\dBm{\dB\of{m}}


\usepackage{lipsum} % Dummytext
\usepackage{blindtext}
\blindmathtrue

 % colors
 \usepackage{color, colortbl}
 %Farbdefinition von hellgrau
 \definecolor{light-gray}{gray}{0.75}
 \definecolor{lightgray}{rgb}{.9,.9,.9}
 \definecolor{darkgray}{rgb}{.4,.4,.4}
 \definecolor{purple}{rgb}{0.65, 0.12, 0.82}
 \definecolor{darkgreen}{RGB}{0, 238, 0}
 \definecolor{blue}{RGB}{44, 111, 235}
 %%%VISUAL STUDIO COLORS
 \definecolor{VSblue}{RGB}{44, 111, 235}
 \definecolor{VSstringred}{RGB}{163, 21, 21}
 \definecolor{VScitrine}{rgb}{0.89, 0.82, 0.04}
 \definecolor{VSblack}{rgb}{0.0, 0.0, 0.0}
 \definecolor{VSgreencomments}{rgb}{0,0.5,0}
 \definecolor{VSclassGreen}{RGB}{0, 125, 154}
 
 \usepackage[dvipsnames,table,xcdraw]{xcolor} % needs to be loaded before pgfplots!


%floats
\usepackage{float}
\addto\extrasngerman{\let\subsectionautorefname\sectionautorefname 	\let\subsubsectionautorefname\sectionautorefname} %Displays sub- and subsubsection as section in \autoref


\usepackage[
	%format=hang, justification=justified,									% format option 1: indented text,	label normal,	separated with colon,	justification
	format=plain, justification=justified, labelfont=bf, labelsep=quad, 	% format option 2: no indents,		label bold,		separated with tab,		justification
	singlelinecheck=true % center captions if they fit into a single line 
	]{caption}
\usepackage{subcaption} % adds caption support for subfigures
%\captionsetup[table]{position=above, aboveskip=4pt}%, belowskip=6pt} % should place the caption of tables above the table. Does not work!

%tables
\usepackage{booktabs} % make nice tables
\usepackage{multirow}
\usepackage{tabularx} % add support for line wrapping in tabels

\newcolumntype{P}[1]{>{\raggedright\arraybackslash}p{#1}}
\newcolumntype{L}{>{$}l<{$}} % math-mode version of "l" column type

\renewcommand{\arraystretch}{2}
\setlength{\tabcolsep}{5mm}
\usepackage{array}% in the preamble

%figures
\usepackage[pdftex]{graphicx}		
\graphicspath{{img/}}
\usepackage{wrapfig} % allow placement of figures in text (wrap text around figure)

\usepackage{flafter} % make sure figures are placed after declaration

% TikZ/PGF

\usepackage{tikz}
\usepackage{pgffor} % foreach loops
\usepackage{pgfplots}
\usepackage{circuitikz}

% externalize pgf plots
\usepgfplotslibrary{external}
\tikzexternalize
\pgfplotsset{
	width=15cm,
	compat=1.18,
	/pgf/number format/.cd,
	use comma,
	1000 sep={ }
}



\usepackage{tcolorbox} % makes colorful boxes (for examples and such)

% listings
\usepackage{listings}
\usepackage{listings-ext}
\lstset{%
	%language=[Sharp]C,			% Standardsprache des Quellcodes
	%backgroundcolor=\color{lightgray}, %Hintergrundfarbe von CODE
	numbers=left,			% Zeilennummern links
	stepnumber=1,			% Jede Zeile nummerieren.
	numbersep=5pt,			% 5pt Abstand zum Quellcode
	numberstyle=\tiny,		% Zeichengrösse 'tiny' für die Nummern.
	breaklines=true,		% Zeilen umbrechen wenn notwendig.
	breakautoindent=true,	% Nach dem Zeilenumbruch Zeile einrücken.
	postbreak=\space,		% Bei Leerzeichen umbrechen.
	tabsize=2,				% Tabulatorgrösse 2
	basicstyle=\ttfamily\footnotesize, % Nichtproportionale Schrift, klein für den Quellcode
	showspaces=false,		% Leerzeichen nicht anzeigen.
	showstringspaces=false,	% Leerzeichen auch in Strings ('') nicht anzeigen.
	extendedchars=true,		% Alle Zeichen vom Latin1 Zeichensatz anzeigen.
	captionpos=b,			% sets the caption-position to bottom
	%backgroundcolor=\color{ListingBackground}, % Hintergrundfarbe des Quellcodes setzen.
	xleftmargin=0pt,		% Rand links
	xrightmargin=0pt,		% Rand rechts
	frame=single,			% Rahmen an
	frameround=ffff,
	rulecolor=\color{darkgray},	% Rahmenfarbe
	%fillcolor=\color{ListingBackground},
	keywordstyle=\color[rgb]{0.133,0.133,0.6},
	commentstyle=\color[rgb]{0.133,0.545,0.133},
	stringstyle=\color[rgb]{0.627,0.126,0.941}
} % uses textcomp
\renewcommand\lstlistingname{Codeausschnitt}
\renewcommand\lstlistlistingname{Quellcodeverzeichnis}

%Umlaute und Sonderzeichen im listings erlauben
\lstset{literate=%
	{Ö}{{\"O}}1
	{Ä}{{\"A}}1
	{Ü}{{\"U}}1
	{ß}{{\ss}}2
	{ü}{{\"u}}1
	{ä}{{\"a}}1
	{ö}{{\"o}}1
	{&}{{\&}}1
}


\usepackage{enumitem} % allows control about enumeration format

% acronyms
\usepackage{acro}

\NewAcroTemplate{shortLong}{%
	\acroiffirstTF{%
		\acrowrite{short}%
		#2%
		\acspace%
		(\ifstrempty{#2}{}{\acrowrite{short}: }%
		\acroifT{foreign}{\acrowrite{foreign}, }%
		\acrowrite{long}%
		\acroifT{alt}{ \acrotranslate{or} \acrowrite{alt}})%
	}%
	{\acrowrite{short}#2}%
}

\RenewAcroCommand\ac{m O{}}{\UseAcroTemplate{shortLong}[2]{#1}{#2}}
\RenewAcroCommand\acp{m O{}}{\acroplural\UseAcroTemplate{shortLong}[2]{#1}{#2}}
\RenewAcroCommand\iac{m O{}}{\acroindefinite\UseAcroTemplate{shortLong}[2]{#1}{#2}}
\RenewAcroCommand\Ac{m O{}}{\acroupper\UseAcroTemplate{shortLong}[2]{#1}{#2}}
\RenewAcroCommand\Acp{m O{}}{\acroplural\acroupper\UseAcroTemplate{shortLong}[2]{#1}{#2}}
\RenewAcroCommand\Iac{m O{}}{\acroupper\acroindefinite\UseAcroTemplate{shortLong}[2]{#1}{#2}}

\acsetup{
	first-style=shortLong,
	trailing/activate={dash}, 
	make-links=true,
	list/heading=addchap, % format table of acronyms (toa) as chapter* and add tot toc (\addchap of KOMA)
	list/name= {Abkürzungsverzeichnis} % make table-of-commands style uniform
}


%bibliography
\usepackage[style=ieee, %authortitle, %Zitierstil
	bibstyle=ieee, %authortitle,
	language=ngerman, %Quellen auf Deutsch
	hyperref=true, %Anklickbare Referenzen
	natbib=true, 
	backend=biber, 
	pagetracker=true, %ebd. bei wiederholten Angaben
	bibencoding=utf8,
	backrefstyle=three+, % fasst Seiten zusammen, z. B S. 2f, 6ff, 7-10
	date=short, %Datumsformat
	bibwarn=true]{biblatex}

\renewcommand*{\labelalphaothers}{\textsuperscript{+}} % handle multiple authors
	
\usepackage[babel,german=guillemets]{csquotes} % select quotes for bibliography
\setlength{\bibitemsep}{1em}     % Abstand zwischen den Literaturangaben
\setlength{\bibhang}{2em}        % Einzug nach jeweils erster Zeile
% include bib files (configured in Einstellungen.tex)
\ladeliteratur

\setcounter{biburllcpenalty}{7000}
\setcounter{biburlucpenalty}{8000}

				
% kopf und fußzeile
\usepackage[headsepline, footsepline, plainheadsepline, plainfootsepline]{scrlayer-scrpage}		% linie aktivieren für "headings" und "plain" seiten
\automark{chapter}		% aktivieren das "chapter" und "section" oeben angezeigt werden kann
\automark*{section}

\setlength{\headheight}{35pt}			% header abstand
\setlength{\footheight}{\baselineskip}	% footer abstand

\clearpairofpagestyles

% Der "*" macht, dass hier konfigurierte Einstellungen auch für "plain"-Seiten gelten
%\ihead*{\vspace*{-0.0cm}\includegraphics[height=1cm, width=0.3\textwidth, keepaspectratio, angle=0]{\firmenlogo}}	% firmenlogo etwas runtersetzen, da komisch formatiert
\chead{\normalfont \headmark}		% "chapter" und "section" oben mittig anzeigen
\ohead*{\includegraphics[height=2cm, angle=0]{img/dhbwlogo.jpg}}	% DHBW Logo

\ifoot*{\scriptsize \normalfont Autor: \autor}	% \scriptsize -> kleine Schriftgröße      \normalfont -> Nicht Kursiv
\cfoot*{\scriptsize \normalfont Datum: \today}  % Autor Links, Datum mitte, Seitenzahl rechts
\ofoot{\pagemark}

\renewcommand*{\chapterpagestyle}{headings}		% pagestyle von der ersten seite eines kapitels ändern
\renewcommand*{\titlepagestyle}{plain}		% pagestyle von der titelseite ändern (funktioniert hier nicht, da kein "richtiger" titel nach koma-skript gesetzt wird)

\renewcaptionname{ngerman}{\bibname}{Literaturverzeichnis} % change name of bibliography

% Formelverzeichniss
\DeclareNewTOC[
	tocentryindent=0pt,
    tocentrynumwidth=2em,
	type=equation,
	name={Gl.},
	types=equations,
	listname={Formelverzeichnis},
]{equ}
\newcommand{\addequations}[2][\theequation]{%
	\addxcontentsline{equ}{equation}[{#1}]{\kern 1em #2}%
}


\usepackage{url} % hyphenate urls - needs to be loaded before hyperref-package!
\usepackage{bookmark} %nur ein latex-Durchlauf für die Aktualisierung von Verzeichnissen nötig

% allow hyper links in pdf
\usepackage{hyperref}
\hypersetup{
	pdftitle={\titel},
	pdfauthor={\autor},
	pdfsubject={\art},
	pdfcreator={pdflatex, LaTeX with KOMA-Script},
	pdfpagemode=UseOutlines, 		% Beim Oeffnen Inhaltsverzeichnis anzeigen
	pdfdisplaydoctitle=true, % Dokumenttitel statt Dateiname anzeigen.
	breaklinks=true		
}



\newenvironment{symboltable}{
	\par
	\let\stretchbuf\arraystretch
	\renewcommand{\arraystretch}{1}
	\tabularx{\textwidth}{rX}
}{
	\endtabularx
	\renewcommand{\arraystretch}{\stretchbuf}
	\par
}