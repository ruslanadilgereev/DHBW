\chapter{Herausforderungen und Lernkurve im Projektverlauf}
Im Verlauf der Projektarbeit traten eine Reihe von praktischen Schwierigkeiten auf, die sowohl technische als auch
konzeptionelle Aspekte betrafen. Zu Beginn gestaltete sich bereits das Einrichten der Entwicklungsumgebung und der 
benötigten Software, insbesondere von MariaDB, als unerwartet aufwändig. Hinzu kam, dass zunächst keinerlei Vorkenntnisse 
in Flutter vorlagen, weshalb eine intensive Einarbeitungsphase erforderlich war, um grundlegende Prinzipien der Flutter-Architektur,
Widgets sowie State-Management zu verstehen und sicher anzuwenden.

Auch auf der UI-Ebene gab es immer wieder komplexe Herausforderungen. Das Darstellen einzelner Interface-Elemente, allen voran die
sogenannte „Training Card“, führte zu unterschiedlichen Fehlerbildern in verschiedenen Ansichten der Anwendung. Parallel dazu 
wuchsen mit fortschreitendem Funktionsumfang die Anforderungen an die Funktionalität: Die Implementierung einer Jahresansicht
für die Trainingskalender, das Hinzufügen effizienter Suchfunktionen und eine flexible Filtermöglichkeit erforderten ein
tiefgreifendes Verständnis der Flutter-Widgets, asynchroner Datenabfragen sowie einer sauberen Trennung von Logik und Darstellung.

Nicht zuletzt stellte auch die Interaktion mit der Datenbank eine anspruchsvolle Aufgabe dar. Der Aufbau performanter,
sicherer und gleichzeitig wartungsfreundlicher Datenbankabfragen sowie die Koppelung an die Frontend-Logik verlangten
sowohl Sorgfalt als auch ein gewisses Maß an Trial-and-Error. Diese Erfahrungen führten letztlich zu einem enormen Lernzuwachs,
stärkten das Verständnis für die gesamtheitliche Systemarchitektur und halfen dabei, künftige Herausforderungen systematischer
und effizienter anzugehen.