\chapter{Ausführen der Software}

Um die Software ausführen zu können, müssen sowohl Flutter als auch Dart installiert sein. Zudem müssen die in der \textit{"pubspec.yaml"}-Datei angegebenen Pakete installiert werden. Dies erfolgt durch Ausführen des folgenden Befehls in der Konsole, während man sich im Projektordner befindet:
\newline
\begin{lstlisting}[caption={Installieren der benötigten Pakete}, label={lst:pubget}]
    flutter pub get
\end{lstlisting}

Für die Einbindung der MariaDB-Datenbank sind möglicherweise Anpassungen in der Datei \textit{"flutter\_backend/.env"} erforderlich, um den Port und den Host korrekt zu konfigurieren.

Zum Starten der App muss zunächst das Backend gestartet werden. Dies geschieht durch Ausführen des folgenden Befehls im Ordner \textit{"flutter\_backend"}:
\newline
\begin{lstlisting}[caption={Starten des Backends}, label={lst:backend}]
    npm run dev
\end{lstlisting}

Um das Frontend zu starten, muss in einer weiteren Konsole der folgende Befehl ausgeführt werden, während man sich im Projektordner befindet. Dadurch wird das Frontend am richtigen Port gestartet:
\newline
\begin{lstlisting}[caption={Starten des Frontends}, label={lst:frontend}]
    flutter run --web-port 3000
\end{lstlisting}
