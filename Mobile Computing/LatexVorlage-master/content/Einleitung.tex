\chapter{Einleitung}

In der folgenden Dokumentation wird die Implementierung einer Kalender-App beschrieben, mit der Trainingseinheiten gebucht
 werden können. Die Dokumentation erläutert, wie diese Anwendung konzipiert und umgesetzt wurde. Aus der Aufgabenstellung ergaben 
 sich die folgenden Vorgaben für die Bearbeitung:

\begin{itemize}
    \item Anbindung einer Datenbank mittels MariaDB,
    \item Implementierung einer Kalenderjahresansicht,
    \item Farbschema bestehend aus den Farben Blau, Grau und Weiß.
\end{itemize}

Zusätzlich zu diesen vorgegebenen Features konnten noch die folgenden Funktionen integriert werden:

\begin{itemize}
    \item Startseite der App mit allgemeinen Informationen und rechtlichen Hinweisen,
    \item weitere Kalenderansichten (Jahr, Monat, Woche),
    \item Darstellung der Trainingseinheiten in einer Listenansicht (alle Trainings, gebuchte Trainings),
    \item Sortierfunktion für die Listenansichten,
    \item E-Mail-Bestätigung bei der Anmeldung,
    \item Implementierung von Anbieter- und Benutzerrollen:
    \begin{itemize}
        \item Anbieter können Trainings und Trainer hinzufügen,
        \item Benutzer können Trainings buchen.
    \end{itemize}
    \item zusätzliches Farbschema im Dark Mode,
    \item Filterfunktionen (z. B. Suche nach Begriffen und Tags).
\end{itemize}
