\chapter{Fazit und Ausblick}

Die vorgestellte Anwendung verdeutlicht ein in sich stimmiges Gesamtkonzept, das auf professionelle Weiterbildung und 
Kursverwaltung ausgerichtet ist. Im Zentrum steht eine mobile Flutter-App, die Nutzerinnen ein ansprechendes, intuitives 
Interface für die Buchung und Verwaltung von Schulungs- und Trainingsangeboten bietet. Die klare Trennung zwischen Frontend,
 Backend und Datenhaltung schafft dabei eine robuste Grundlage, um reibungslose Abläufe und hohe Performanz sicherzustellen. 
 Nutzerinnen können Angebote durchsuchen, gezielt filtern, Termine buchen oder stornieren und erhalten durch die Integration 
 eines E-Mail-Services automatisierte Benachrichtigungen, etwa zu bevorstehenden Kursen oder erfolgreichen Anmeldungen.
Die Architektur ist so gestaltet, dass sie nicht nur die unmittelbaren Anforderungen erfüllt, sondern auch langfristige 
Weiterentwicklung ermöglicht. Dank sauber definierter Schnittstellen und eines klaren State-Managements lassen sich neue 
Funktionen, zusätzliche Kurskategorien oder weiterführende Analysewerkzeuge ohne grundsätzliche Umbrüche integrieren.
 Das Backend sorgt über transparente APIs und effektives Session-Management für Datensicherheit und Verlässlichkeit. 
 Die relationale MySQL-Datenbank ermöglicht es, selbst komplexe Kursstrukturen, Terminpläne, Pausenregelungen und Nutzerprofile 
 zentral abzubilden, was eine einfache Auswertung, Erweiterung und Pflege der Inhalte gestattet.
Insgesamt zeigt sich, dass die Applikation nicht nur eine aktuelle Lösung für ein bestimmtes Anwendungsfeld darstellt,
sondern ein solides Fundament für künftige Anforderungen liefert. Die modulare und gut dokumentierte Architektur trägt
maßgeblich zur Wartungsfreundlichkeit, Skalierbarkeit und Flexibilität bei. Damit ist die Anwendung ideal darauf vorbereitet,
mit den Erwartungen ihrer Nutzer*innen mitzuwachsen und langfristig ein verlässlicher Begleiter im Schulungs-
und Weiterbildungsbereich zu sein.
