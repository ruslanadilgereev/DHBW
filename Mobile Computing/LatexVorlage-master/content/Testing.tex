\chapter{Testing}

Um die Kalender-App zu testen, wurden zwei Arten von Tests durchgeführt: Widget-Tests und Unit-Tests.

\section{Unit-Test}

Mithilfe eines Unit-Tests können einzelne Methoden und Funktionen überprüft werden. Im Test werden die erforderlichen 
Eingabeparameter simuliert und die Ausgabeparameter auf ihre Korrektheit hin überprüft. Ein Unit-Test greift während der 
Testdurchführung nicht auf die Festplatte zu. Dadurch werden weder Grafiken gerendert noch Benutzereingaben berücksichtigt.

Unit-Tests dienen daher ausschließlich dazu, die Logik einzelner Methoden zu überprüfen. Sie eignen sich jedoch nicht, um die
 Schnittstellenfunktionalität oder die Interaktion mit Benutzern zu testen.

\section{Widget-Test}
Der Widget-Test wird verwendet, um die Interaktion mit dem Benutzer und die grafische Darstellung der App zu überprüfen. In einer 
isolierten Testumgebung werden Widgets erzeugt und getestet, ob sie korrekt gerendert wurden. Darüber hinaus wird überprüft, 
ob Benutzereingaben korrekt verarbeitet werden und das erneute Rendering der Benutzeroberfläche ordnungsgemäß funktioniert.

\section{Implementierung}

Es wurden zwei Arten von Tests implementiert, um die Funktionalität der Kalender-App sicherzustellen:
\begin{itemize}
    \item Unit-Tests, um die Logik einzelner Methoden innerhalb von Flutter zu überprüfen.
    \item Widget-Tests, um die Benutzerinteraktion und die korrekte grafische Darstellung der Anwendung zu validieren.
\end{itemize}

Der Code für die Testprogramme wurde in den folgenden Dateien umgesetzt: \textit{"training\_service\_test.dart"},
 \textit{"training\_service\_test.mocks.dart"} und \textit{"widget\_test.dart"}.
Bei \textit{"training\_service\_test.dart"} und der Datei \textit{"training\_service\_test.mocks.dart"} handelt es sich
um den Unit-Test, wobei die Datei \textit{"training\_service\_test.mocks.dart"} eine Simulationsumgebung umsetzt und die andere Datei
 den eigentlichen Test enthält.
Durch die begrenzte Zeit wurde jeder Test jeweils für eine Methode implementiert. Mittels des Unit-Tests wurde die Funktion der
 Trainingssuche getestet. Dabei wurden die folgenden vier Szenarien betrachtet:
\begin{itemize}
    \item Erfolgreiche Antworten (gültige Trainingsdaten).
    \item Fehlerhafte Antworten (Fehlerstatus oder ungültige JSON-Daten).
    \item Leere Antworten (keine Trainings gefunden).
    \item Formatierung von null-Werten oder fehlenden Daten.
\end{itemize}

Beim Widget-Test wurde die ordnungsgemäße Initialisierung der \textit{ThemeService}- und \textit{AuthService}-Provider getestet. 
Das Ergebnis des Testprotokolls befindet sich im Anhang.
